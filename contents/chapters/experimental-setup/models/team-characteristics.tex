\subsection{Team characteristics}

This model uses the ideas from \cref{subsec:variables-team-characteristics}. The model also incorporates some of the ideas from \citet{bib:constantinou-fenton-neil-2012}, whereas teams are evaluated anonymously. Teams are represented as the set of their characteristics at match start.

The rationale behind this model is to capture what makes teams win against some opponents, but lose against others. Aside from team strengths and player abilities, what decides a football match? The idea is that the model will capture cases where one team has characteristics that give an advantage over the other team's characteristics. For example if a strength of the home team is "Creating long shot opportunities", whilst a weakness of the away team is "Defending against long shots".

\subsubsection{Input}

Each team is represented as $43$ values in the range $[0, 1]$. Each value corresponds to a team characteristic (see: \cref{subsec:database-overview-team-stats}). Values corresponding to present styles are set to $1$. Strengths and weaknesses are divided by $100$ to fit them into the range $[0, 0.55]$. With $43$ values for each team, that gives a total of $43 * 2 = 86$ features for each match.

For the match between Arsenal and Manchester United May 7, 2017, the input values would be as follows:
\begin{lstlisting}[language=Python]
    [1.0, 1.0, 1.0, 1.0, 0.0, 1.0, 0.0, 0.0, 0.0, 0.0, 0.0, 0.55, 0.55, 0.45, 0.45, 0.0, 0.0, 0.0, 0.25, 0.0, 0.0, 0.0, 0.0, 0.0, 0.0, 0.0, 0.0, 0.0, 0.0, 0.0, 0.15, 0.0, 0.0, 0.0, 0.0, 0.45, 0.0, 0.0, 0.0, 0.45, 1.0, 0.0, 1.0, 1.0, 1.0, 1.0, 0.0, 0.0, 0.0, 1.0, 0.0, 0.0, 1.0, 0.0, 0.25, 0.0, 0.45, 0.0, 0.45, 0.55, 0.45, 0.15, 0.0, 0.0, 0.45, 0.0, 0.0, 0.0, 0.0, 0.0, 0.0, 0.0, 0.0, 0.0, 0.0, 0.0, 0.0, 0.0, 0.55, 0.0, 0.0, 1.0, 0.55, 0.0, 1.0, 1.0]
\end{lstlisting}
The first 43 values are Arsenal characteristics. The last 43 are Manchester United characteristics.