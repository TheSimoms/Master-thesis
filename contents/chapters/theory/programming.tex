\section{Regular expressions}

A regular expression is a string describing a set of strings, a pattern, following given syntax rules. Regular expressions are widely used in text editors, used when searching and replacing text based on patterns. Regular expressions are supported by several programming languages, such as JavaScript and Python.

An important part of regular expressions are meta-characters. Meta-characters describe a set of characters, and allow evaluating logical expressions on the string.

\iffalse
        \^, \$                  & Start and end of string \\\hline
        .                       & Any character except newline \\\hline
        \textbackslash .        & Escaping of a meta-character (in this case \textit{.}) \\\hline
        \[abc\]                 & Group of characters (in this case, \textit{a}, \textit{b}, or \textit{c} \\\hline
        a\|b                    & Logical or of characters (in this case \textit{a} or \textit{b} \\\hline
        \textbackslash s, \textbackslash d, \textbackslash w           & Whitespace character, digit, and word character (\[a-zA-Z0-9\_\]) \\\hline
        \{m, n\}                & Between \textit{m} and \textit{n} inclusive occurrences of the previous meta-character \\\hline
        ?                       & Zero or one occurrences of the previous meta-character \\\hline
        +                       & One or more occurrences of the previous meta-character \\\hline
        *                       & Zero or more occurrences of the previous meta-character \\\hline
\fi



\section{HTML}

\gls{html} is the de facto standard markup language for creating web pages. A \gls{html} document is built up by tags describing how the web page should look and behave. Web browsers read \gls{html} documents and present the content accordingly.


\section{JavaScript}

JavaScript is a high-level programming language. Together with \gls{html}, JavaScript is a powerful tool supported by just about every modern web browser. JavaScript is versatile, supporting both object oriented-programming and functional programming.

JavaScript is natively supported in \gls{html}, and is extremely useful for creating dynamic web pages. A popular use case for JavaScript is dynamic loading of data without reloading the web page.


\section{Python}

Python is a high-level programming language. It supports several programming paradigms, such as imperative, functional, object-oriented, and procedural programming. Python was originally developed as a scripting language, but is today used is several other applications, such as web servers and desktop applications.

There is a vast community around Python, and countless libraries for everything from web scraping to 3D animation.


\section{Tensorflow}

"TensorFlow\texttrademark is an open source software library for numerical computation using data flow graphs. Nodes in the graph represent mathematical operations, while the graph edges represent the multidimensional data arrays (tensors) communicated between them. The flexible architecture allows you to deploy computation to one or more CPUs or GPUs in a desktop, server, or mobile device with a single API. TensorFlow was originally developed by researchers and engineers working on the Google Brain Team within Google's Machine Intelligence research organization for the purposes of conducting machine learning and deep neural networks research, but the system is general enough to be applicable in a wide variety of other domains as well" \citep{bib:tensorflow}.


\section{Keras}

Keras is a high-level \gls{ann} API, written in Python. Keras does not implement any form for \gls{ann} functionality itself, but runs on top other libraries, such as TensorFlow. Keras was developed with a focus on user friendliness, modularity, and easy extensibility \citep{bib:keras}.