\restylefloat{figure}


\newcommand{\HRule}{\rule{\linewidth}{0.5mm}}


\renewcommand{\chaptermark}[1]{\markboth{\chaptername\ \thechapter.\ #1}{}}
\renewcommand{\sectionmark}[1]{\markright{\thesection\ #1}}
\renewcommand{\headrulewidth}{0.1ex}
\renewcommand{\footrulewidth}{0.1ex}


\fancypagestyle{plain}{\fancyhf{}\fancyfoot[LE,RO]{\thepage}\renewcommand{\headrulewidth}{0ex}}


% Document information
\title{Beating the bookmakers}
\author{Simon Borøy-Johnsen}


% Make sure floats are placed correctly
\let\Oldsection\section
\renewcommand{\section}{\FloatBarrier\Oldsection}

\let\Oldsubsection\subsection
\renewcommand{\subsection}{\FloatBarrier\Oldsubsection}

\let\Oldsubsubsection\subsubsection
\renewcommand{\subsubsection}{\FloatBarrier\Oldsubsubsection}

\floatplacement{figure}{ht}
\floatplacement{table}{ht}


% Rename appendix chapter to appendix
\crefname{appendix}{Appendix}{Appendices}


% Custom date
\newdateformat{monthyear}{\monthname[\THEMONTH] \THEYEAR}


% Rename lists
\renewcommand*\contentsname{Table of Contents}
\renewcommand*{\acronymname}{List of Abbreviations}
\renewcommand*{\lstlistlistingname}{List of Listings}


% Add list of listings to toc
\renewcommand{\lstlistoflistings}{\begingroup\tocfile{\lstlistlistingname}{lol}\endgroup}


% Generate abbreviations
\newacronym{ai}{AI}{Artificial Intelligence}

\newacronym{idi}{IDI}{Department of Computer Science}
\newacronym{ntnu}{NTNU}{Norwegian University of Science and Technology}

\newacronym{ann}{ANN}{Artificial Neural Network}
\newacronym{fnn}{FNN}{Feedforward Neural Network}
\newacronym{rnn}{RNN}{Recurrent Neural Network}
\newacronym{cnn}{CNN}{Convolutional Neural Network}

\newacronym{relu}{ReLU}{Rectified Linear Unit}
\newacronym{tanh}{tanh}{Hyperbolic Tangent}

\newacronym{knn}{kNN}{k-Nearest Neighbor}
\newacronym{svm}{SVM}{Support Vector Machine}

\newacronym{mse}{MSE}{Mean Squared Error}
\newacronym{mae}{MAE}{Mean Absolute Error}
\newacronym{rps}{RPS}{Rank Probability Score}
\newacronym{roi}{ROI}{Return On Investment}

\newacronym{html}{HTML}{HyperText Markup Language}
\newacronym{sql}{SQL}{Structured Query Language}

\makeglossaries


% Add prefix to figures
\graphicspath{ {figures/} }


% Adding internal counters to figures and tables
\counterwithin{figure}{section}
\counterwithin{table}{section}


% Add bibliography
\addbibresource{references.bib}


% Set code listings style
\lstset{basicstyle=\ttfamily}
\lstset{breaklines}


% Add bozes around algorithms
\RestyleAlgo{boxruled}


% Set numbering depth
\setcounter{tocdepth}{1} % Includes up to sections in the table of contents
\setcounter{secnumdepth}{2} % Numbers up to sub-subsections


% Check-mark and cross
\newcommand{\cmark}{\ding{51}}
\newcommand{\xmark}{\ding{55}}


% Other custom commands
\newcommand{\whoscoredplain}{www.whoscored.com}
\newcommand{\whoscored}{\textbf{\whoscoredplain}}
\newcommand{\whoscoredurl}[1]{\url{\whoscoredplain#1}}


% Nicer parentheses in equations
\delimitershortfall = -1pt


% Best configuration color in results
\definecolor{correct}{rgb}{0.56, 0.93, 0.56}


% For JavaScript listings
\definecolor{lightgray}{rgb}{.9,.9,.9}
\definecolor{darkgray}{rgb}{.4,.4,.4}
\definecolor{purple}{rgb}{0.65, 0.12, 0.82}

\lstdefinelanguage{JavaScript}{
  keywords={break, case, catch, continue, debugger, default, delete, do, else, false, finally, for, function, if, in, instanceof, new, null, return, switch, this, throw, true, try, typeof, var, void, while, with},
  morecomment=[l]{//},
  morecomment=[s]{/*}{*/},
  morestring=[b]',
  morestring=[b]",
  ndkeywords={class, export, boolean, throw, implements, import, this},
  keywordstyle=\color{blue}\bfseries,
  ndkeywordstyle=\color{darkgray}\bfseries,
  identifierstyle=\color{black},
  commentstyle=\color{purple}\ttfamily,
  stringstyle=\color{red}\ttfamily,
  sensitive=true
}

\lstset{
   language=JavaScript,
   backgroundcolor=\color{lightgray},
   extendedchars=true,
   basicstyle=\footnotesize\ttfamily,
   showstringspaces=false,
   showspaces=false,
   numbers=left,
   numberstyle=\footnotesize,
   numbersep=9pt,
   tabsize=2,
   breaklines=true,
   showtabs=false,
   captionpos=b
}


% Setting up tikz

\usetikzlibrary{positioning}

\tikzset{%
  every neuron/.style={
    circle,
    draw,
    minimum size=17pt
  },
  neuron missing/.style={
    draw=none, 
    scale=2,
    text height=0.333cm,
    execute at begin node=\color{black}$\vdots$
  },
}