% Add prefix to figures
\graphicspath{ {figures/} }


% Adding internal counters to figures and tables
\counterwithin{figure}{section}
\counterwithin{table}{section}


% Custom date
\newdateformat{monthyear}{\monthname[\THEMONTH] \THEYEAR}


% Add bibliography
\addbibresource{references.bib}


% Set code listings style
\lstset{basicstyle=\ttfamily}
\lstset{breaklines}


% Refer to figure, table, or equation
\newcommand{\referfigure}[1]{\hyperref[#1]{Figure \ref*{#1}}}
\newcommand{\refertable}[1]{\hyperref[#1]{Table \ref*{#1}}}
\newcommand{\referequation}[1]{\hyperref[#1]{Equation \ref*{#1}}}
\newcommand{\refersection}[1]{\hyperref[#1]{Section \ref*{#1}}}
\newcommand{\referchapter}[1]{\hyperref[#1]{Chapter \ref*{#1}}}
\newcommand{\referappendix}[1]{\hyperref[#1]{Appendix \ref*{#1}}}


% Set numbering depth
\setcounter{tocdepth}{1} %Includes up to sections in the table of contents
\setcounter{secnumdepth}{3} % Numbers up to sub-subsections


% Load acronyms
%\newacronym{}{}{}
\newacronym{ann}{ANN}{artificial neural network}
\newacronym{fnn}{FNN}{feedforward neural network}
\newacronym{rnn}{RNN}{recurrent neural network}
\newacronym{cnn}{CNN}{convolutional neural network}

\newacronym{knn}{kNN}{k-nearest neighbor}
\newacronym{svm}{SVM}{support vector machine}

\newacronym{mse}{MSE}{mean squared error}
\newacronym{mae}{MAE}{mean absolute error}

\newacronym{html}{HTML}{HyperText Markup Language}


% Check-mark and cross
\newcommand{\cmark}{\ding{51}}
\newcommand{\xmark}{\ding{55}}


% Other custom commands
\newcommand{\whoscoredplain}{www.whoscored.com}
\newcommand{\whoscored}{\textbf{\whoscoredplain}}
\newcommand{\whoscoredurl}[1]{\url{\whoscoredplain#1}}


% Nicer parentheses in equations
\delimitershortfall = -1pt

% For JavaScript listings
\definecolor{lightgray}{rgb}{.9,.9,.9}
\definecolor{darkgray}{rgb}{.4,.4,.4}
\definecolor{purple}{rgb}{0.65, 0.12, 0.82}

\lstdefinelanguage{JavaScript}{
  keywords={break, case, catch, continue, debugger, default, delete, do, else, false, finally, for, function, if, in, instanceof, new, null, return, switch, this, throw, true, try, typeof, var, void, while, with},
  morecomment=[l]{//},
  morecomment=[s]{/*}{*/},
  morestring=[b]',
  morestring=[b]",
  ndkeywords={class, export, boolean, throw, implements, import, this},
  keywordstyle=\color{blue}\bfseries,
  ndkeywordstyle=\color{darkgray}\bfseries,
  identifierstyle=\color{black},
  commentstyle=\color{purple}\ttfamily,
  stringstyle=\color{red}\ttfamily,
  sensitive=true
}

\lstset{
   language=JavaScript,
   backgroundcolor=\color{lightgray},
   extendedchars=true,
   basicstyle=\footnotesize\ttfamily,
   showstringspaces=false,
   showspaces=false,
   numbers=left,
   numberstyle=\footnotesize,
   numbersep=9pt,
   tabsize=2,
   breaklines=true,
   showtabs=false,
   captionpos=b
}